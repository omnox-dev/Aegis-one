\documentclass[a4paper,12pt]{article}
\usepackage{geometry}
\usepackage{graphicx}
\usepackage{hyperref}
\usepackage{listings}
\usepackage{color}
\usepackage{float}
\usepackage{svg}
\usepackage{longtable}
\usepackage{caption}

\geometry{margin=1in}

\title{\textbf{AEGIS One Project Documentation}}
\author{Development Team}
\date{\today}

\begin{document}

\maketitle

\tableofcontents
\newpage

\section{Introduction}

\subsection{Project Overview}
**AEGIS One** is a unified, comprehensive digital platform designed to modernize university campus management. It serves as a central hub for academic tracking, administrative processes, student engagement, and emergency response, eliminating the inefficiency of disjointed systems.

\subsection{Problem Context}
University ecosystems are fragmented. Students often navigate multiple portals for grades, resort to informal chat groups for official announcements, lack a transparent channel for grievances, and miss out on collaborative opportunities due to poor visibility.

This fragmentation leads to:
\begin{itemize}
    \item Missed academic deadlines and club events.
    \item Untracked and unresolved student grievances.
    \item Safety concerns due to slow emergency reporting.
    \item Inefficient resource usage (e.g., empty seats in cabs to nearby cities).
\end{itemize}

\section{System Architecture \& Technology Stack}

\subsection{Architecture Diagram}
The application employs a decoupled client-server architecture to ensure scalability and maintainability.

\begin{figure}[H]
    \centering
    \includegraphics[width=1.0\textwidth]{diagrams/arch-aegisone.png}
    \caption{High-Level System Architecture}
\end{figure}

\subsection{Technology Stack & Justification}
The platform is built on a robust, industry-standard stack selected for performance and scalability.

\begin{description}
    \item[Frontend: Next.js (React)] selected for its server-side rendering capabilities (SEO), performance optimization, and the modern App Router for intuitive navigation.
    \item[Styling: Tailwind CSS] chosen for rapid UI development and a consistent, utility-first design system that ensures mobile responsiveness out of the box.
    \item[Backend: FastAPI (Python)] utilized for its high performance (async/await), automatic interactive API documentation (Swagger UI), and ease of integration with Python-based data science libraries.
    \item[Database: PostgreSQL] preferred for its reliability, ACID compliance, and robust support for complex relational data models required in university management.
    \item[ORM: SQLAlchemy] ensures type-safe database interactions and simplifies schema migrations.
    \item[Authentication: JWT + OAuth2] implements a stateless, secure, and scalable authentication standard suitable for modern web apps.
\end{description}

\section{Feature Pillars}

\subsection{Core Pillars (Essential Modules)}
\begin{enumerate}
    \item \textbf{Academic Hub}: Centralized dashboard for course enrollments, syllabus tracking, attendance monitoring, and resource sharing (notes/PYQs).
    \item \textbf{Grievance Redressal}: A transparent system for logging complaints (Hostel, Mess, Academic) with status tracking (Pending $\rightarrow$ In Review $\rightarrow$ Resolved).
    \item \textbf{Club Management}: Discovery portal for student organizations, event calendars, and membership management.
    \item \textbf{Identity & Access}: Role-based access control (RBAC) ensuring appropriate permissions for Students, Faculty, and Admin users.
\end{enumerate}

\subsection{Bonus Pillars (Advanced Features)}
\begin{enumerate}
    \item \textbf{Caravan (Smart Carpooling)}: A sustainable travel module allowing students to pool rides to nearby transit hubs, splitting costs and reducing carbon footprint.
    \item \textbf{Mercenary (Student Gig Economy)}: A peer-to-peer marketplace where students can offer skills (booking tutors, designers, coders) for bounties.
    \item \textbf{SOS / Emergency Response}: One-tap emergency trigger that shares live location with campus security authorities.
    \item \textbf{Lost \& Found}: A digital board for reporting lost items or claiming found ones, complete with image uploads.
    \item \textbf{Community Forum}: Integrated discussion boards for polling, academic doubts, and campus-wide discussions.
\end{enumerate}

\section{Database Design}

The database schema supports 18 distinct entities, covering all aspects of student life.

\subsection{Entity-Relationship Diagram (ERD)}
\begin{figure}[H]
    \centering
    \includesvg[width=1.0\textwidth]{diagrams/DB-relation.svg}
    \caption{Database Schema Overview}
\end{figure}

\section{UML Diagrams}

\subsection{Class Diagram}
\begin{figure}[H]
    \centering
    \includegraphics[width=0.8\textwidth]{diagrams/class.png}
    \caption{Backend Class Structure}
\end{figure}

\section{Installation \& Setup Guide}

\subsection{Prerequisites}
\begin{itemize}
    \item Python 3.10+, Node.js 18+
    \item PostgreSQL Database
\end{itemize}

\subsection{Step-by-Step Installation}
\begin{enumerate}
    \item \textbf{Clone Repository}:
    \begin{lstlisting}[language=bash, basicstyle=\small\ttfamily, frame=single]
git clone https://github.com/omnox-dev/Aegis-one.git
    \end{lstlisting}
    
    \item \textbf{Backend Setup}:
    \begin{lstlisting}[language=bash, basicstyle=\small\ttfamily, frame=single]
cd backend
python -m venv venv
source venv/bin/activate  # Windows: venv\Scripts\activate
pip install -r requirements.txt
# Configure .env file with DATABASE_URL
alembic upgrade head
uvicorn app.main:app --reload
    \end{lstlisting}
    
    \item \textbf{Frontend Setup}:
    \begin{lstlisting}[language=bash, basicstyle=\small\ttfamily, frame=single]
cd frontend
npm install
npm run dev
    \end{lstlisting}
\end{enumerate}

\section{User Interface Showcase}
\textit{(Screenshots to be added here)}

\begin{figure}[H]
    \centering
    \fbox{
        \begin{minipage}{0.45\textwidth}
            \centering
            \vspace{3cm} \textbf{[Dashboard Screenshot]} \vspace{3cm}
        \end{minipage}
    }
    \hfill
    \fbox{
        \begin{minipage}{0.45\textwidth}
            \centering
            \vspace{3cm} \textbf{[Grievance Portal Screenshot]} \vspace{3cm}
        \end{minipage}
    }
    \caption{Student Dashboard & Grievance Portal}
\end{figure}

\section{Conclusion \& Future Scope}

\subsection{Known Limitations}
\begin{itemize}
    \item \textbf{Payment Gateway}: Financial transactions in the Mercenary module are currently manual.
    \item \textbf{Real-Time Chat}: Messaging relies on REST polling rather than WebSockets.
\end{itemize}

\subsection{Future Roadmap}
\begin{itemize}
    \item AI-driven recommendations for clubs and internships.
    \item Mobile application (React Native) for on-the-go access.
    \item Blockchain-based certificate verification for internships.
\end{itemize}

\end{document}
